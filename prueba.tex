\documentclass{article}
\usepackage[utf8]{inputenc}
\usepackage{amsmath}
\usepackage[spanish]{babel}
\title{Apuntes de Programación lineal}
\author{Reyna Edgar}

\begin{document}

\maketitle
\tableofcontents
\section{Introducción}
\label{sec:inicio}

La forma estándar de un problema de programación lineal es:
Dada una matriz $A$ y vectores $b$ y $C$ maximizar $C^TX$ sujeto a
$AX\leq b$ y $x\geq0$.

\smallskip
Si alguna de nuestras variables no está acotada por 0 hacemos un cambio de variable y resolvemos el problema nuevo.

La forma simplex de un problema de programación líneal es:
Dada una matrix $A$ y vectores $c$ y $b$. Maximizar $c^Tx$ sujeto a
$Ax=b$ y $x\geq0$.

\section{Ejemplo}
\label{sec:ejemplo-1}

Un gerente está planeando como distribuir la producción de dos productos entre dos máquinas. Para ser manufacturado cada producto requiere cierto tiempo (en horas) en cada una de las máquinas.
El tiempo requerido está resumido en la siguiente tabla:
\begin{center}
  \begin{tabular}
  {|c|c|c|}
  \hline
  &A&B\\
  \hline
  máquina&1&2\\
  \hline
  máquina&1 &1\\
  \hline
\end{tabular}
\end{center}

La máquina 1 está disponible 40 horas a la semana y la 2 está disponible 34 horas a la semana.

Si la utilidad obtenida al vender los productos A y B es de 2, 3 pesos por unidad, respectivamente ¿Cuál debe ser la producción semanal que maximiza la utilidad? ¿Cuál es la utilidad máxima?

\begin{equation*}
  \label{eq:1}
  \begin{pmatrix}
    0&1&2\\
    3&-1&5
  \end{pmatrix}
  \begin{pmatrix}
    9&2\\
    7&8\\
    5&3\\
  \end{pmatrix}
\end{equation*}
\end{document}